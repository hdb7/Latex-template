\documentclass{beamer}
\usepackage{amsmath}
\usepackage{amssymb}
\usepackage{mathrsfs}
\usepackage{graphicx} %for inserting images
\graphicspath{ {./images/} }  % folder where images file is
\usepackage[autostyle]{csquotes}
%\usetheme{Warsaw}
%\usetheme{CambridgeUS}
%\usetheme{Madrid}
\usetheme{Pittsburgh}
\title{Partial Order Set}
\subtitle{}
\author{Hamjak Debbarma}
\institute[TIT]{
Dept. Of Computer Science\\
Tripura Institute of Technology, Agartala\\
\medskip
\textit{hamjakdb@yahoo.com}
}
\date{\today}

\begin{document}

\begin{frame}
\titlepage
\end{frame}

\begin{frame}
\frametitle{Outline}
\begin{itemize}
\item What is POSET ? \pause
\item Proof some results \pause
\item Product Order
\end{itemize}
\end{frame}

\begin{frame}
\frametitle{What is POSET ?}
Let $R\subseteq A\times A$, where $A$ is finite set. Then, $R$ is partial order relation iff -
\begin{itemize}
\item Reflexive \(\forall a\in A, (a,a)\in R\)
\item Anti-symmetric \(\forall a,b\in A, (a,b)\in R\; \text{and}\; (b,a)\in R \implies a=b\)
\item Transitive \(\forall a,b,c\in A, (a,b)\in R\; \text{and}\; (b,c)\in R\) implies, $(a,c)\in R$.
\end{itemize}
\end{frame}

\begin{frame}
\frametitle{POSET}
\begin{definition}
A set $S$ together with the partial order relation $R$ is called a \textit{partial order set or POSET}. Denoted as $(S,\preceq)$.
\end{definition}
\textbf{Note: } $\preceq$ is not $\leq$, its read as \enquote{precedes}.\\
\textbf{Examples,} $a \preceq b$ read as $a$ `precedes' $b$
\end{frame}

\begin{frame}
\begin{example}
Q. Show that $R=\{(a,b)\in \mathbb{Z}\times\mathbb{Z} \mid a-b \leq 0\}$ is partial order.
\end{example}
\begin{block}{Proof}
Let $a\in \mathbb{Z}$ then, $a-a \leq 0$. Therefore, $(a,a)\in R, \forall a\in \mathbb{Z}$ so $R$ is reflexive. Now, for $(a,b)\in R$ and  $(b,a)\in R$, $a-b \leq 0$ and $b-a \leq 0$ which implies, $a\leq b$ and $b\leq a \implies a=b$, $R$ is Anti-symmetric. Again, for $(a,b)\in R$ and  $(b,c)\in R$, $a-b \leq 0$ and $b-a \leq 0$ implies, $a-b+b-c \leq 0 \implies a-c \leq 0$.Therefore, $R$ is Transitive. Hence, $R$ is partial order.
\end{block}
\end{frame}

\begin{frame}
\begin{example}
Q. Is $R=\{(m,n)\in \mathbb{Z}\times \mathbb{Z} \mid m+n\; \text{is even}\; \forall m,n \in \mathbb{Z}\}$ a partial order?
\end{example}
\begin{block}{Proof}
We know $m$ and $n$ has to be both odd or even for $m+n$ to be even, $m$ and $n$ do not have to be necessarily same for their sum to be even. So, it is not Anti-symmetric. Hence, $R$ is not partial order relation.
\end{block}
\end{frame}

\begin{frame}
\frametitle{Product Order}
\begin{definition}
Let $s,s^{'}\in S$ and $t,t^{'}\in T$, $s\preceq s^{'}$ in $S$ and $t\preceq t^{'}$ in $T$ then, $S\times T$ is called the \textbf{product order} such that $(s,t)\preceq (s^{'},t^{'})\in S\times T$
\end{definition}
\end{frame}

\begin{frame}
\begin{theorem}
If $(A,\preceq)$ and $(B,\preceq)$ are poset, then $(A\times B, \preceq)$ is also poset, where $\preceq$ is defined by $(a,b)\preceq (a^{'},b^{'})$ if $a\preceq a^{'}$ in $A$ and $b\preceq b^{'}$ in $B$.
\end{theorem}
\end{frame}

\begin{frame}
\begin{block}{Summary}
\begin{itemize}
\item We have learn what a POSET is ?
\item we have seen when $R$ is partial order
\end{itemize}
\end{block}
\end{frame}
\end{document}